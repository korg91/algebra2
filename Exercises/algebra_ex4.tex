\documentclass[12pt,a4paper]{report}

\usepackage{amsmath}
\usepackage{bbm}
\usepackage[utf8]{inputenc}
\usepackage{longtable}
\usepackage{amsthm}
\usepackage{amscd}
\usepackage{amssymb}
\usepackage{amsfonts}
\usepackage{amsmath}
\usepackage{mathtools}
\usepackage{enumitem}
\usepackage[hyphens]{url}
\usepackage[scale=3]{ccicons}  % per le icone creative commons
\usepackage{hyperref}  % per i link nel pdf
\usepackage[rmargin=3.0cm,lmargin=3.0cm]{geometry}
%\usepackage{frontesp}  % prima pagina; il pacchetto frontesp.sty si trova nella stessa cartella del file .tex (deve essere adattato a mano)
\usepackage{setspace}  % per l'interlinea
\usepackage[english]{babel}  % per sillabazione
\usepackage[all]{xy} %diagrammi di funzioni
\usepackage{xspace} %per assicurare la corretta gestione degli spazi finali quando uso e.g. \AC. NB: sarebbe meglio trovare un'altra soluzione...cfr. http://tex.stackexchange.com/questions/15220/no-space-present-after-ensuremath



\theoremstyle{definition}
\newtheorem{teo}{Teorema}[section]  % resetta la numerazione dei teoremi per ogni capitolo
\newtheorem{defn}[teo]{Definizione}  % la numerazione delle definizioni dipende da quella dei teoremi
\newtheorem{es}[teo]{Esempio}  % idem
\newtheorem{oss}[teo]{Osservazione}  % idem
\newtheorem{prop}[teo]{Proposizione}  % idem
\newtheorem{lemma}[teo]{Lemma}  % idem
\newtheorem{corollario}[teo]{Corollario}  % idem

%%% inizio comandi per stile per teoremi: "numero. Titolo" %%%
\newtheoremstyle{num.custom-title}
  {\topsep}   % ABOVESPACE
  {\topsep}   % BELOWSPACE
  {\normalfont}  % BODYFONT
  {0pt}       % INDENT (empty value is the same as 0pt)
  {\bfseries} % HEADFONT
  {}         % HEADPUNCT
  {5pt plus 1pt minus 1pt} % HEADSPACE
  {\thmnumber{#2.}\thmnote{ #3}}
  
\theoremstyle{num.custom-title}  
\newtheorem{teo_custom-title}[teo]{} % per usarlo basta \begin{teo_custom-title}[<Titolo teorema>] (usa automaticamente la numerazione di [teo])
%%% fine comandi per stile per teoremi: "numero. Titolo" %%%


\DeclareMathOperator{\dom}{dom}
\DeclareMathOperator{\ran}{ran}
\DeclareMathOperator{\orb}{orb}
\DeclareMathOperator{\id}{id}
\DeclareMathOperator{\rk}{rk}
\DeclareMathOperator{\tor}{tor}
\DeclareMathOperator{\Zdv}{Zdv}
\DeclareMathOperator{\Hom}{Hom}
\DeclareMathOperator{\End}{End}
\DeclareMathOperator{\Ann}{Ann}
\DeclareMathOperator{\A}{\mathcal{A}}
\DeclareMathOperator{\B}{\mathcal{B}}
\DeclareMathOperator{\PP}{\mathcal{P}}
\DeclareMathOperator{\LL}{\mathcal{L}}
\DeclareMathOperator{\Hrtg}{\text{Hrtg}}
\DeclareMathOperator{\Ord}{\text{Ord}}
\DeclareMathOperator{\N}{\mathbb{N}}
\DeclareMathOperator{\Q}{\mathbb{Q}}
\DeclareMathOperator{\R}{\mathbb{R}}
\DeclareMathOperator{\Z}{\mathbb{Z}}
\DeclareMathOperator{\M}{\mathfrak{M}}
\DeclareMathOperator{\U}{\mathfrak{U}}
\DeclareMathOperator{\PPP}{\mathbb{P}}
\DeclareMathOperator{\a01}{\{0,1\}^{\star}}
\DeclareMathOperator{\imp}{\Rightarrow}
\DeclareMathOperator{\sm}{\setminus}
\DeclareMathOperator{\sse}{\subseteq}


\newcommand{\AC}{\ensuremath{\mathsf{AC}}\xspace}
\newcommand{\CC}{\ensuremath{\mathsf{CC}}\xspace}
\newcommand{\DC}{\ensuremath{\mathsf{DC}}\xspace}
\newcommand{\ZF}{\ensuremath{\mathsf{ZF}}\xspace}
\newcommand{\ZFC}{\ensuremath{\mathsf{ZFC}}\xspace}
\newcommand{\LS}{\ensuremath{\mathsf{LS}}\xspace}
\newcommand{\AMC}{\ensuremath{\mathsf{AMC}}\xspace}
\newcommand{\HRule}{\rule{\linewidth}{0.5mm}} %per la prima pagina

\renewcommand{\phi}{\varphi}
\renewcommand{\S}{\mathcal{S}}


%%%% INIZIO COMANDI PER EQUIVALENZE %%%%
\newcommand{\Implies}[2]{$\text{\ref{statement#1}}\!\implies\!\text{\ref{statement#2}}$}% X => Y
\newcommand{\punto}[1]{\item \label{statement#1}}


\newenvironment{equivalence}
    {\begin{enumerate}[label=(\arabic*),ref=(\arabic*)]
    }
    { 
	\end{enumerate}
    }
%%%% FINE COMANDI PER EQUIVALENZE %%%



% Interlinea 1.5
%\onehalfspacing  


%per le citazioni
\def\signed #1{{\leavevmode\unskip\nobreak\hfil\penalty50\hskip2em
  \hbox{}\nobreak\hfil(#1)%
  \parfillskip=0pt \finalhyphendemerits=0 \endgraf}}

\newsavebox\mybox
\newenvironment{aquote}[1]
  {\savebox\mybox{#1}\begin{quote}}
  {\signed{\usebox\mybox}\end{quote}}

%disabilita colore link
\hypersetup{%
    pdfborder = {0 0 0}
}

\begin{document}

\paragraph{Exercise 21.} 
\begin{proof}
Let $M=\langle x_1, ..., x_n \rangle$. Consider the morphism $\phi: R \to M^n$ given by $r \mapsto (r x_1, ...., r x_n)$. It is immediate to check that $\ker \phi = \Ann_R M$. So $\phi(R) \simeq R/\Ann_R M$. But $\phi(R)$ is a submodule of $M^k$, which is noetherian by theory. Thus $\phi(R)$ is noetherian as well, and we are done.
\end{proof}


\paragraph{Exercise 22.}
\begin{proof}\ 
\begin{enumerate}
\item Let $M_{\tor} = \{x \in M \mid \Ann_R (x) \neq 0\}$. Let $x,y \in M_{\tor}$. Then there exist $\lambda, \mu \in R$ s.t. $\lambda x = \mu x = 0$. Thus $\lambda \mu (x+y) =0$, i.e. $x+y \in M_{\tor}$. The closure under scalar multiplication is trivial.
\item Suppose $M= \langle x_1,...,x_n \rangle$.  By hypothesis, there exist $\lambda_1,...,\lambda_n \in R$ s.t. $\lambda_i x_i = 0$ for all $i$. Thus, for every $x \in M$, $(\lambda_1 \cdot ... \cdot \lambda_n) x= \lambda_1 \cdot ... \cdot \lambda_n (a_1 x_1 + ... + a_n x_n)=0$.
\item Let $(x_1,...,x_n)$ be a basis for $M$. Let $0 \neq x = a_1 x_1 + ... + a_n x_n$. Suppose $0 = \lambda x = \lambda (a_1 x_1 + ... + a_n x_n)$. Since $(x_1,...,x_n)$ is a basis, this means $\lambda a_i = 0$ for all $a_i$. By hypothesis, $x \neq 0$, so at least one $a_k$ is not $0$. Since $R$ is a domain, we have $\lambda = 0 \vee a_k = 0$, thus $\lambda=0$. MANCA IL CONTROESEMPIO!!!
\end{enumerate}
\end{proof}

\paragraph{Exercise 23.}
\begin{proof}
First of all, we prove that $\{a \in S \mid aS \sse R\}= \Ann_R(S/R)$. The inclusion $\sse$ is trivial consequence of the definition of the factor ring. For the inclusion $\supseteq$, observe that $a (x + R) = 0 \imp ax + R = 0 \imp ax \in R$.\\
It is immediate to show that $\mathfrak{f}_{S/R}$ is an ideal of $S$ and $R$. Consider now an ideal $I$ of $S$ which is also an ideal of $R$. Since every ideal is closed under multiplication of elements of the ring, we have that, for all $a \in I$ and $x \in S$, $a x \in I \sse R$, thus $a S \sse R$. That is, $I \sse \mathfrak{f}_{S/R}$.
\end{proof}

\paragraph{Exercise 24.}
\begin{proof}
Suppose $S= \langle x_1,...,x_n \rangle$. Since $S \sse q(R)$, we must have $x_i = a_i b_i^{-1}$ for some $a,b \in R$. Consider an arbitrary $x \in S$, that is $x = \lambda_1 a_1 b_1^{-1} + ... + \lambda_n a_n b_n^{-1}$, with $\lambda_1,...,\lambda_n \in R$. Then obviously $(b_1 \cdot ... \cdot b_n) x \in R$. Thus $b_1 \cdot ... \cdot b_n \in \mathfrak{f}_{S/R}$.
\end{proof}

\paragraph{Exercise 26.} 
\begin{proof}
Follows trivially by Corollary 2.36.
\end{proof}



\end{document}