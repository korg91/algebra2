\documentclass[12pt,a4paper]{report}

\usepackage{amsmath}
\usepackage{bbm}
\usepackage[utf8]{inputenc}
\usepackage{longtable}
\usepackage{amsthm}
\usepackage{amscd}
\usepackage{amssymb}
\usepackage{amsfonts}
\usepackage{amsmath}
\usepackage{mathtools}
\usepackage{enumitem}
\usepackage[hyphens]{url}
\usepackage[scale=3]{ccicons}  % per le icone creative commons
\usepackage{hyperref}  % per i link nel pdf
\usepackage[rmargin=3.0cm,lmargin=3.0cm]{geometry}
%\usepackage{frontesp}  % prima pagina; il pacchetto frontesp.sty si trova nella stessa cartella del file .tex (deve essere adattato a mano)
\usepackage{setspace}  % per l'interlinea
\usepackage[english]{babel}  % per sillabazione
\usepackage[all]{xy} %diagrammi di funzioni
\usepackage{xspace} %per assicurare la corretta gestione degli spazi finali quando uso e.g. \AC. NB: sarebbe meglio trovare un'altra soluzione...cfr. http://tex.stackexchange.com/questions/15220/no-space-present-after-ensuremath



\theoremstyle{definition}
\newtheorem{theorem}{Theorem}[chapter] % resetta la numerazione dei teoremi per ogni capitolo
\newtheorem{corollary}[theorem]{Corollary} % la numerazione delle definizioni dipende da quella dei teoremi
\newtheorem{lemma}[theorem]{Lemma}
\newtheorem{defn}[theorem]{Definition}
\newtheorem*{addendum}{Addendum}
\newtheorem*{remark}{Remark}

%%% inizio comandi per stile per teoremi: "numero. Titolo" %%%
\newtheoremstyle{num.custom-title}
  {\topsep}   % ABOVESPACE
  {\topsep}   % BELOWSPACE
  {\normalfont}  % BODYFONT
  {0pt}       % INDENT (empty value is the same as 0pt)
  {\bfseries} % HEADFONT
  {}         % HEADPUNCT
  {5pt plus 1pt minus 1pt} % HEADSPACE
  {\thmnumber{#2.}\thmnote{ #3}}
  
\theoremstyle{num.custom-title}  
\newtheorem{teo_custom-title}[theorem]{} % per usarlo basta \begin{teo_custom-title}[<Titolo teorema>] (usa automaticamente la numerazione di [teo])
%%% fine comandi per stile per teoremi: "numero. Titolo" %%%

\newenvironment{claim}[1]{\par\noindent\underline{Claim#1:}\space}{} %per i claim
\newenvironment{claimproof}[1]{\par\noindent\underline{Proof:}\space#1}{\leavevmode\unskip\penalty9999 \hbox{}\nobreak\hfill\quad\hbox{$\blacksquare$}} %per le dimostrazioni dei claim

\DeclareMathOperator{\dom}{dom}
\DeclareMathOperator{\ran}{ran}
\DeclareMathOperator{\orb}{orb}
\DeclareMathOperator{\id}{id}
\DeclareMathOperator{\rk}{rk}
\DeclareMathOperator{\tor}{tor}
\DeclareMathOperator{\Zdv}{Zdv}
\DeclareMathOperator{\Hom}{Hom}
\DeclareMathOperator{\End}{End}
\DeclareMathOperator{\Ann}{Ann}
\DeclareMathOperator{\A}{\mathcal{A}}
\DeclareMathOperator{\B}{\mathcal{B}}
\DeclareMathOperator{\PP}{\mathcal{P}}
\DeclareMathOperator{\LL}{\mathcal{L}}
\DeclareMathOperator{\Hrtg}{\text{Hrtg}}
\DeclareMathOperator{\Ord}{\text{Ord}}
\DeclareMathOperator{\J}{\mathcal{J}}
\DeclareMathOperator{\N}{\mathbb{N}}
\DeclareMathOperator{\Q}{\mathbb{Q}}
\DeclareMathOperator{\R}{\mathbb{R}}
\DeclareMathOperator{\Z}{\mathbb{Z}}
\DeclareMathOperator{\M}{\mathfrak{M}}
\DeclareMathOperator{\U}{\mathfrak{U}}
\DeclareMathOperator{\PPP}{\mathbb{P}}
\DeclareMathOperator{\p}{\mathfrak{p}}
\DeclareMathOperator{\V}{\mathcal{V}}
\DeclareMathOperator{\a01}{\{0,1\}^{\star}}
\DeclareMathOperator{\imp}{\Rightarrow}
\DeclareMathOperator{\sm}{\setminus}
\DeclareMathOperator{\sse}{\subseteq}
\DeclareMathOperator{\Spec}{Spec}


\newcommand{\AC}{\ensuremath{\mathsf{AC}}\xspace}
\newcommand{\CC}{\ensuremath{\mathsf{CC}}\xspace}
\newcommand{\DC}{\ensuremath{\mathsf{DC}}\xspace}
\newcommand{\ZF}{\ensuremath{\mathsf{ZF}}\xspace}
\newcommand{\ZFC}{\ensuremath{\mathsf{ZFC}}\xspace}
\newcommand{\LS}{\ensuremath{\mathsf{LS}}\xspace}
\newcommand{\AMC}{\ensuremath{\mathsf{AMC}}\xspace}
\newcommand{\HRule}{\rule{\linewidth}{0.5mm}} %per la prima pagina
\newcommand{\qedblack}{\hfill $\blacksquare$}
\newcommand{\ol}{\overline}
\newcommand{\ul}{\underline}

\renewcommand{\phi}{\varphi}
\renewcommand{\S}{\mathcal{S}}
\renewcommand{\P}{\mathbb{P}}


%%%% INIZIO COMANDI PER EQUIVALENZE %%%%
\newcommand{\Implies}[2]{$\text{\ref{statement#1}}\!\implies\!\text{\ref{statement#2}}$}% X => Y
\newcommand{\punto}[1]{\item \label{statement#1}}


\newenvironment{equivalence}
    {\begin{enumerate}[label=(\arabic*),ref=(\arabic*)]
    }
    { 
	\end{enumerate}
    }
%%%% FINE COMANDI PER EQUIVALENZE %%%



% Interlinea 1.5
%\onehalfspacing  


%per le citazioni
\def\signed #1{{\leavevmode\unskip\nobreak\hfil\penalty50\hskip2em
  \hbox{}\nobreak\hfil(#1)%
  \parfillskip=0pt \finalhyphendemerits=0 \endgraf}}

\newsavebox\mybox
\newenvironment{aquote}[1]
  {\savebox\mybox{#1}\begin{quote}}
  {\signed{\usebox\mybox}\end{quote}}

%disabilita colore link
\hypersetup{%
    pdfborder = {0 0 0}
}

\begin{document}

\noindent Andrea Gadotti \hfill 01/12/2014

\paragraph{Exercise 48.}
\begin{enumerate}
\item We have to check that $I$ is closed under addition and multiplication by elements of $R$. The closure under addition is trivial since $I$ is an ideal. The closure under multiplication by elements of $R$ follows by the following:
\[
\lambda \in R, x \in I \imp \lambda x = \lambda (1_A x) = (\lambda 1_A) x
\]
where the last item is in $I$ since $\lambda 1_A$ is an element of $A$ and thus $(\lambda 1_A) x \in I$ since $I$ is an ideal of $A$.
\item $A$ is a noetherian $R$-module because it is surjective image of $\oplus_{i=1}^n R$ for some $n \in \N$ (trivial), which is noetherian since $R$ is noetherian, and by theory we know that finite direct sum of noetherian modules is noetherian and that surjective image of a noetherian module is noetherian. We want now to prove that $A$ is a noetherian ring, that is, $A$ is a noetherian $A$-module. The $A$-submodules of $A$ are the ideals of $A$. But thanks to the first point, these are also $R$-modules, thus any ascending chain of them must stabilize.
\end{enumerate}

\paragraph{Exercise 49.} $\Z$ is a reduced ring (every integral domain is trivially a reduced ring).
\begin{enumerate}
\item Let $[x] \in R/I$ such that $[x]^n=0$ for some $n \in \N$. This means $x^n \in I$, but since $I$ is radical we have $x \in I$, that is $[x]=0$.
\item Thanks to the first point, it suffices to prove that $\mathcal{I}(V)$ is radical. But this has already been proven in the proof of Hilbert's Nullstellenstatz (proof: If $f \in K[\mathbf{X}]$ with $f^n \in \mathcal{I}(V)$, then $f^n(\ul{a})=0$ for all $\ul{a} \in V$, and thus $f(\ul{a})=0$ for all $\ul{a} \in V$, since $L$ is a field and thus a domain).
\end{enumerate}

\paragraph{Exercise 50.} 
\begin{enumerate}
\item We know by Proposition 3.23 that
\[
I=\sqrt{I}= \bigcap_{\p \in \nu(I)} \p,
\]
where $\nu(I):=\{\p \in \Spec(I) \mid \p \supseteq I\}$. But by Theorem 3.18 (whose proof seems to work even getting rid of the ``domain'' hypothesis) we know that every noetherian ring has only a finite number of prime ideals. Therefore $\nu(I)$ is finite, and we are done.
\item Of course we have that 
\[
\p_1 \cap ... \cap \p_r \sse \mathfrak{q}_k
\]
for all $k=1..s$. By Theorem 3.4, this means that $\p_i \sse \mathfrak{q}_k$ for some $i=1..r$. Repeating the same argument, we obtain that for all $i=1..r$ there exists $k=1..s$ such that $\mathfrak{q}_k \sse \p_i$. We must have that the equality $\p_i=\mathfrak{q}_k$ holds. In fact, suppose to the contrary that $\p_i \sse \mathfrak{q}_k$ and $\mathfrak{q}_{k'} \sse \p_i$. Then $\mathfrak{q}_{k'} \sse \mathfrak{q}_k$, which contradicts the hypothesis. So we are done.
\end{enumerate}

\paragraph{Exercise 51.}\ \\
(a)$\imp$(b): By contraposition, suppose there exist $f,g \not\in \J(V)$ with $fg \in \J(V)$. Consider $\J(V)+(f) \supsetneq \J(V)$ and $\J(V)+(g) \supsetneq \J(V)$, and define $V_1:=\V_L(\J(V)+(f))$, $V_2:=\V_L(\J(V)+(g))$. Of course $V_1,V_2 \subsetneq V$. But $V=V_1 \cup V_2$, because (since $fg \in \J(V)$) we have $x \in V \imp fg(x)=0 \imp f(x)g(x)=0 \imp f(x)=0$ or $g(x)=0 \imp x \in V_1$ or $x \in V_2$.\\
(b)$\imp$(a): By contraposition, suppose $V=V_1 \cup V_2$ with $V_1,V_2 \neq V$ varieties. Then of course\footnote{If $\J(V)=\J(V_1)$, then $V=\V_L(\J(V))=\V_L(\J(V_1))=V_1$.} $\J(V) \subsetneq \J(V_1)$ and $\J(V) \subsetneq \J(V_2)$, that is, there exist $f \in \J(V_1) \sm \J(V)$ and $g \in \J(V_2) \sm \J(V)$. But $fg$ vanishes on $V=V_1 \cup V_2$, so $fg \in \J(V)$, i.e. $\J(V)$ is not prime.




\paragraph{Exercise 52.} 
\begin{enumerate}
\item TODO
\item We have $\J(V)=\J(V_1 \cup ... \cup V_s=\J(V_1) \cap ... \cap \J(V_s)$. $\J(V_i)$ is a prime ideal for all $i=1..s$ thanks to Exercise 51, and thus $\J(V_i)=\p_i$ (after renumbering if necessary) thanks to Exercise 50.2. Therefore $V_i=\V_L(\J(V_i))=\V_L(\p_i)$.
\end{enumerate}

\end{document}