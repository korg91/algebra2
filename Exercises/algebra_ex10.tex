\documentclass[12pt,a4paper]{report}

\usepackage{amsmath}
\usepackage{bbm}
\usepackage[utf8]{inputenc}
\usepackage{longtable}
\usepackage{amsthm}
\usepackage{amscd}
\usepackage{amssymb}
\usepackage{amsfonts}
\usepackage{amsmath}
\usepackage{mathtools}
\usepackage{enumitem}
\usepackage[hyphens]{url}
\usepackage[scale=3]{ccicons}  % per le icone creative commons
\usepackage{hyperref}  % per i link nel pdf
\usepackage[rmargin=3.0cm,lmargin=3.0cm]{geometry}
%\usepackage{frontesp}  % prima pagina; il pacchetto frontesp.sty si trova nella stessa cartella del file .tex (deve essere adattato a mano)
\usepackage{setspace}  % per l'interlinea
\usepackage[english]{babel}  % per sillabazione
\usepackage[all]{xy} %diagrammi di funzioni
\usepackage{xspace} %per assicurare la corretta gestione degli spazi finali quando uso e.g. \AC. NB: sarebbe meglio trovare un'altra soluzione...cfr. http://tex.stackexchange.com/questions/15220/no-space-present-after-ensuremath



\theoremstyle{definition}
\newtheorem{theorem}{Theorem}[chapter] % resetta la numerazione dei teoremi per ogni capitolo
\newtheorem{corollary}[theorem]{Corollary} % la numerazione delle definizioni dipende da quella dei teoremi
\newtheorem{lemma}[theorem]{Lemma}
\newtheorem{defn}[theorem]{Definition}
\newtheorem*{addendum}{Addendum}
\newtheorem*{remark}{Remark}

%%% inizio comandi per stile per teoremi: "numero. Titolo" %%%
\newtheoremstyle{num.custom-title}
  {\topsep}   % ABOVESPACE
  {\topsep}   % BELOWSPACE
  {\normalfont}  % BODYFONT
  {0pt}       % INDENT (empty value is the same as 0pt)
  {\bfseries} % HEADFONT
  {}         % HEADPUNCT
  {5pt plus 1pt minus 1pt} % HEADSPACE
  {\thmnumber{#2.}\thmnote{ #3}}
  
\theoremstyle{num.custom-title}  
\newtheorem{teo_custom-title}[theorem]{} % per usarlo basta \begin{teo_custom-title}[<Titolo teorema>] (usa automaticamente la numerazione di [teo])
%%% fine comandi per stile per teoremi: "numero. Titolo" %%%

\newenvironment{claim}[1]{\par\noindent\underline{Claim#1:}\space}{} %per i claim
\newenvironment{claimproof}[1]{\par\noindent\underline{Proof:}\space#1}{\leavevmode\unskip\penalty9999 \hbox{}\nobreak\hfill\quad\hbox{$\blacksquare$}} %per le dimostrazioni dei claim

\DeclareMathOperator{\dom}{dom}
\DeclareMathOperator{\ran}{ran}
\DeclareMathOperator{\orb}{orb}
\DeclareMathOperator{\id}{id}
\DeclareMathOperator{\rk}{rk}
\DeclareMathOperator{\tor}{tor}
\DeclareMathOperator{\Zdv}{Zdv}
\DeclareMathOperator{\Hom}{Hom}
\DeclareMathOperator{\End}{End}
\DeclareMathOperator{\Ann}{Ann}
\DeclareMathOperator{\A}{\mathcal{A}}
\DeclareMathOperator{\B}{\mathcal{B}}
\DeclareMathOperator{\PP}{\mathcal{P}}
\DeclareMathOperator{\LL}{\mathcal{L}}
\DeclareMathOperator{\Hrtg}{\text{Hrtg}}
\DeclareMathOperator{\Ord}{\text{Ord}}
\DeclareMathOperator{\J}{\mathcal{J}}
\DeclareMathOperator{\N}{\mathbb{N}}
\DeclareMathOperator{\Q}{\mathbb{Q}}
\DeclareMathOperator{\R}{\mathbb{R}}
\DeclareMathOperator{\Z}{\mathbb{Z}}
\DeclareMathOperator{\M}{\mathfrak{M}}
\DeclareMathOperator{\U}{\mathfrak{U}}
\DeclareMathOperator{\PPP}{\mathbb{P}}
\DeclareMathOperator{\p}{\mathfrak{p}}
\DeclareMathOperator{\V}{\mathcal{V}}
\DeclareMathOperator{\a01}{\{0,1\}^{\star}}
\DeclareMathOperator{\imp}{\Rightarrow}
\DeclareMathOperator{\sm}{\setminus}
\DeclareMathOperator{\sse}{\subseteq}
\DeclareMathOperator{\Spec}{Spec}


\newcommand{\AC}{\ensuremath{\mathsf{AC}}\xspace}
\newcommand{\CC}{\ensuremath{\mathsf{CC}}\xspace}
\newcommand{\DC}{\ensuremath{\mathsf{DC}}\xspace}
\newcommand{\ZF}{\ensuremath{\mathsf{ZF}}\xspace}
\newcommand{\ZFC}{\ensuremath{\mathsf{ZFC}}\xspace}
\newcommand{\LS}{\ensuremath{\mathsf{LS}}\xspace}
\newcommand{\AMC}{\ensuremath{\mathsf{AMC}}\xspace}
\newcommand{\HRule}{\rule{\linewidth}{0.5mm}} %per la prima pagina
\newcommand{\qedblack}{\hfill $\blacksquare$}
\newcommand{\ol}{\overline}
\newcommand{\ul}{\underline}
\newcommand{\q}{\mathsf{q}}

\renewcommand{\phi}{\varphi}
\renewcommand{\S}{\mathcal{S}}
\renewcommand{\P}{\mathbb{P}}


%%%% INIZIO COMANDI PER EQUIVALENZE %%%%
\newcommand{\Implies}[2]{$\text{\ref{statement#1}}\!\implies\!\text{\ref{statement#2}}$}% X => Y
\newcommand{\punto}[1]{\item \label{statement#1}}


\newenvironment{equivalence}
    {\begin{enumerate}[label=(\arabic*),ref=(\arabic*)]
    }
    { 
	\end{enumerate}
    }
%%%% FINE COMANDI PER EQUIVALENZE %%%



% Interlinea 1.5
%\onehalfspacing  


%per le citazioni
\def\signed #1{{\leavevmode\unskip\nobreak\hfil\penalty50\hskip2em
  \hbox{}\nobreak\hfil(#1)%
  \parfillskip=0pt \finalhyphendemerits=0 \endgraf}}

\newsavebox\mybox
\newenvironment{aquote}[1]
  {\savebox\mybox{#1}\begin{quote}}
  {\signed{\usebox\mybox}\end{quote}}

%disabilita colore link
\hypersetup{%
    pdfborder = {0 0 0}
}

\begin{document}

\noindent Andrea Gadotti \hfill 08/12/2014

\paragraph{Exercise 53.} Let $R$ be a UFD. Consider $x=c^{-1}b \in \q(R)$. Suppose that there exists a polynomial of $R[X]$ which has $x$ as a solution. This means
\[
x^n+a_{n-1}x^{n-1}+ ... +a_0=0
\]
for some $a_0,...,a_{n-1}\in R$. Since $R$ is a UFD, we can suppose that $b,c$ have no non-unit common divisor. Multiply the above equation by $c^n$ to get
\[
b^n+a_{n-1}cb^{n-1}+ ... +a_0b^n=0
\]
i.e.
\[
b^n= -a_{n-1}cb^{n-1}- ... -a_0b^n
\]
Let now $d$ be an irreducible divisor of $c$. Then $d$ is prime since $R$ is a UFD. Now, $d|b^n$ since it divides the member on the right and thus (since $d$ is prime) $d|b$. But $b, c$ have no non-unit common divisors, so $d$ must be a unit. Thus $c$ is a unit as well and hence $x \in R$, as wanted.

\paragraph{Exercise 55.} We follow the hint: let $R_1=K[X]$. So $K[X,Y]=R_1[Y]$. Observe that, since $K$ is a field, $K[X]$ is a PID, and thus a UFD. Thanks to Exercise 53, we have that $R_1=K[X]$ is integrally closed. 
%Furthermore, if $f$ and $g$ are relatively prime in $K[X,Y]$ they must obviously (``even more'') be relatively prime in $R_1[Y]$ as well (that is, they are relatively prime if seen as polynomials over $R_1$ with variable $Y$). MI SA CHE NON SERVE A UN CAZZO STA PRECISAZIONE...LA NOZIONE DI DEFINIBILITÀ È ESATTAMENTE LA STESSA, NON DIPENDE DA COME GUARDO K[X,Y] (se lo guardo come anello dei polinomi in due variabili o come "1 variabile + 1 variabile")
Thus we satisfy the hypothesis of Exercise 54, and so there exist $p,q \in R_1[Y]$ s.t. $pf+qg=a \in R_1=K[X]$. Now suppose $(\alpha,\beta) \in \V_{\ol{K}}(f) \cap \V_{\ol{K}}(g)$. This implies
\[
0=p(\alpha)f(\alpha)+q(\alpha)g(\alpha)=a(\alpha),
\]
i.e. $\alpha$ must be a zero of $a$. By the Foundamental Theorem of Algebra, this means that we have only finitely many choices for $\alpha$.\\
The same argument works defining $R_2=K[Y]$ and considering $K[X,Y]=R_2[X]$. Hence there are only finitely many choices for $\beta$ as well, whereby there are only finitely many choices for $(\alpha,\beta)$.


\paragraph{Exercise 56.} First, observe that we can see $R/(Q \cap R)$ as a subring of $S/Q$ because
\[
\phi: R \to S/Q, \quad r \mapsto r+Q
\]
is a homomorphism and $\ker \phi = Q \cap R$. Now, let $x+Q \in S/Q$. Since $x \in S$, by hypothesis there exist $a_0,...,a_{n-1} \in R$ s.t.
\[
x^n + a_{n-1}x^{n-1}+ ... + a_0 = 0.
\]
We trivially have (in $S/Q$)
\[
(x+Q)^n + (a_{n-1} + Q)(x+Q)^{n-1} + ... + (a_0 + Q) = 0+Q,
\]
so
\[
X^n + (a_{n-1} + Q)X^{n-1} + ... + (a_0 + Q)
\]
is a polinomial in $S/Q[X]$ which has $x$ as a zero. But since $a_0,...,a_{n-1} \in R$, we have that $a_0+Q,...,a_{n-1}+Q \in \phi[R]$, i.e. we can see them as elements of $R/(Q \cap R)$, which means that the polynomial is actually in $R/(Q \cap R)[X]$.
\end{document}