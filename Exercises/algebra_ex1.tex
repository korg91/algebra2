\documentclass[12pt,a4paper]{report}

\usepackage{amsmath}
\usepackage{bbm}
\usepackage[utf8]{inputenc}
\usepackage[italian]{babel}
\usepackage{longtable}
\usepackage{amsthm}
\usepackage{amscd}
\usepackage{amssymb}
\usepackage{amsfonts}
\usepackage{amsmath}
\usepackage{mathtools}
\usepackage{enumitem}
\usepackage[hyphens]{url}
\usepackage[scale=3]{ccicons}  % per le icone creative commons
\usepackage{hyperref}  % per i link nel pdf
\usepackage[rmargin=3.0cm,lmargin=3.0cm]{geometry}
%\usepackage{frontesp}  % prima pagina; il pacchetto frontesp.sty si trova nella stessa cartella del file .tex (deve essere adattato a mano)
\usepackage{setspace}  % per l'interlinea
\usepackage[italian]{babel}  % per sillabazione
\usepackage[all]{xy} %diagrammi di funzioni
\usepackage{xspace} %per assicurare la corretta gestione degli spazi finali quando uso e.g. \AC. NB: sarebbe meglio trovare un'altra soluzione...cfr. http://tex.stackexchange.com/questions/15220/no-space-present-after-ensuremath



\theoremstyle{definition}
\newtheorem{teo}{Teorema}[section]  % resetta la numerazione dei teoremi per ogni capitolo
\newtheorem{defn}[teo]{Definizione}  % la numerazione delle definizioni dipende da quella dei teoremi
\newtheorem{es}[teo]{Esempio}  % idem
\newtheorem{oss}[teo]{Osservazione}  % idem
\newtheorem{prop}[teo]{Proposizione}  % idem
\newtheorem{lemma}[teo]{Lemma}  % idem
\newtheorem{corollario}[teo]{Corollario}  % idem

%%% inizio comandi per stile per teoremi: "numero. Titolo" %%%
\newtheoremstyle{num.custom-title}
  {\topsep}   % ABOVESPACE
  {\topsep}   % BELOWSPACE
  {\normalfont}  % BODYFONT
  {0pt}       % INDENT (empty value is the same as 0pt)
  {\bfseries} % HEADFONT
  {}         % HEADPUNCT
  {5pt plus 1pt minus 1pt} % HEADSPACE
  {\thmnumber{#2.}\thmnote{ #3}}
  
\theoremstyle{num.custom-title}  
\newtheorem{teo_custom-title}[teo]{} % per usarlo basta \begin{teo_custom-title}[<Titolo teorema>] (usa automaticamente la numerazione di [teo])
%%% fine comandi per stile per teoremi: "numero. Titolo" %%%


\DeclareMathOperator{\dom}{dom}
\DeclareMathOperator{\ran}{ran}
\DeclareMathOperator{\orb}{orb}
\DeclareMathOperator{\id}{id}
\DeclareMathOperator{\zdv}{Zdv}
\DeclareMathOperator{\Hom}{Hom}
\DeclareMathOperator{\A}{\mathcal{A}}
\DeclareMathOperator{\B}{\mathcal{B}}
\DeclareMathOperator{\PP}{\mathcal{P}}
\DeclareMathOperator{\LL}{\mathcal{L}}
\DeclareMathOperator{\Hrtg}{\text{Hrtg}}
\DeclareMathOperator{\Ord}{\text{Ord}}
\DeclareMathOperator{\N}{\mathbb{N}}
\DeclareMathOperator{\R}{\mathbb{R}}
\DeclareMathOperator{\Z}{\mathbb{Z}}
\DeclareMathOperator{\M}{\mathfrak{M}}
\DeclareMathOperator{\U}{\mathfrak{U}}
\DeclareMathOperator{\PPP}{\mathbb{P}}
\DeclareMathOperator{\a01}{\{0,1\}^{\star}}
\DeclareMathOperator{\imp}{\Rightarrow}
\DeclareMathOperator{\sm}{\setminus}
\DeclareMathOperator{\sse}{\subseteq}


\newcommand{\AC}{\ensuremath{\mathsf{AC}}\xspace}
\newcommand{\CC}{\ensuremath{\mathsf{CC}}\xspace}
\newcommand{\DC}{\ensuremath{\mathsf{DC}}\xspace}
\newcommand{\ZF}{\ensuremath{\mathsf{ZF}}\xspace}
\newcommand{\ZFC}{\ensuremath{\mathsf{ZFC}}\xspace}
\newcommand{\LS}{\ensuremath{\mathsf{LS}}\xspace}
\newcommand{\AMC}{\ensuremath{\mathsf{AMC}}\xspace}
\newcommand{\HRule}{\rule{\linewidth}{0.5mm}} %per la prima pagina

\renewcommand{\phi}{\varphi}
\renewcommand{\S}{\mathcal{S}}


%%%% INIZIO COMANDI PER EQUIVALENZE %%%%
\newcommand{\Implies}[2]{$\text{\ref{statement#1}}\!\implies\!\text{\ref{statement#2}}$}% X => Y
\newcommand{\punto}[1]{\item \label{statement#1}}


\newenvironment{equivalence}
    {\begin{enumerate}[label=(\arabic*),ref=(\arabic*)]
    }
    { 
	\end{enumerate}
    }
%%%% FINE COMANDI PER EQUIVALENZE %%%



% Interlinea 1.5
%\onehalfspacing  


%per le citazioni
\def\signed #1{{\leavevmode\unskip\nobreak\hfil\penalty50\hskip2em
  \hbox{}\nobreak\hfil(#1)%
  \parfillskip=0pt \finalhyphendemerits=0 \endgraf}}

\newsavebox\mybox
\newenvironment{aquote}[1]
  {\savebox\mybox{#1}\begin{quote}}
  {\signed{\usebox\mybox}\end{quote}}

%disabilita colore link
\hypersetup{%
    pdfborder = {0 0 0}
}

\begin{document}


\paragraph{Exercise 1.} Consider the ring $\Z_n$, with $n \geq 2$.
\begin{itemize}
\item $\Z_n^*=\{[x] \in \Z_n : (x,n)=1\}$ \footnote{It is immediate to check that the set is well-defined.}. In fact $(x,n)=1 \iff \exists a,b \in \Z \big( a x + b n = 1 \big) \iff \exists a,b \in \Z \big( [a x + b n] = [1] \big) \iff \exists a,b \in \Z \big( [a x] + [b n] = [1] \big) \iff \exists a,b \in \Z \big( [a][x] + [b][n] = [1] \big) \iff \exists [a] \in \Z_n \big( [a][x] = 1 \big)$.
\item $\zdv(\Z_n)=(\Z_n^*)^C$. It is sufficient to show $\supseteq$. If $(x,n)=d \neq 1$ then $x \frac{n}{d}= \frac{x}{d} n = cn$ for some $c \in \Z$, i.e. $[x][\frac{n}{d}]=[0]$, and since $d \neq 1 \imp \frac{n}{d} < n \imp [\frac{n}{d}] \neq 0$ we are done.
\end{itemize}

\paragraph{Exercise 2.} Let $R$ be an euclidean domain and $f: R \sm \{0\} \to \N$ a norm on $R$. Let $I \triangleleft R$ an ideal. It is sufficient to show $I \sse gR$ for some $g$. Let $g \in I \sm \{0\}$ be such that $f(g)=\min f[I]$. Let $a \in I$. By hypothesis we have $a=g q + r$ for some $q,r \in R$ such that either $r=0$ or $f(r)<f(g)$. But then $r=a - g q \in I$, thus $r=0$ necessarily by minimality. Therefore $a=g q$, and we are done.

\paragraph{Exercise 3.} Let $I$ be an ideal of a ring $R$. Let $\mathbb L_I$ be the set of all ideals of $R$ which contain $I$. Let $\left({\mathbb L \left({R / I}\right), \subseteq}\right)$ be the set of all ideals of $R / I$. Let the mapping $\Phi_I: \left({\mathbb L_I, \subseteq}\right) \to \left({\mathbb L \left({R / I}\right), \subseteq}\right)$ be defined as:
\[
\forall a \in \mathbb L_I: \Phi_I \left({a}\right) = \pi\left({a}\right)
\]
where $q: a \to a / J$ is the canonical epimorphism from $a$ to $a / J$ from the definition of quotient ring.
Then $\Phi_I$ is an isomorphism.
\begin{proof}
Let $b \in \mathbb L_I$. Of course, $I \subseteq b$. Thus $\pi^{-1} \left({\pi \left({b}\right)}\right) = b + J = b$. Furthermore, let $c$ be an ideal of $R / I$. Then $\pi \left({\pi^{-1} \left({c}\right)}\right) = c$.
Thus $\Phi_I$ is a bijection, and we have that $\forall c \in \mathbb L \left({R / I}\right) \big( \pi^{-1} \left({\Phi_I}\right) c = \pi^{-1} \left({c}\right) \big)$.\\
Now to show that $\Phi_I$ is an isomorphism, let $b_1, b_2 \in \mathbb L_I$. If $b_1 \subseteq b_2$, then $\pi \left({b_1}\right) \subseteq \pi \left({b_2}\right)$.\\
Conversely, suppose $\pi \left({b_1}\right) \subseteq \pi \left({b_2}\right)$. By what we have just proved, $b_1 = \pi^{-1} \left({\pi \left({b_1}\right)}\right) \subseteq  \pi^{-1} \left({\pi \left({b_2}\right)}\right) = b_2$.\\
Thus $\Phi_J$ is an isomorphism.
\end{proof}

\paragraph{Exercise 4.}
Let $\mathcal F$ be the set of ideals of $R$ of the form $xR$, with $x$ not a unit and such that $x$ cannot be decomposed in the form: $x = u p_1 \cdots p_r$
with $u$ a unit and $p_1,\ldots,p_r$ irreducible. We show towards a contradiction that $\mathcal F = \emptyset$. Suppose $\mathcal F \neq \emptyset$. Since $R$ is noetherian, we can choose a maximal element $a R \in \mathcal F$. By construction, $a$ is not irreducible, so we can write $a = bc$ with $b,c$ non-units and not associates. Since $a$ and $b$ are not associate, we have $bR \subsetneq aR$ and $aR \subsetneq bR$ (??????????). Since $aR$ is assumed maximal, this means that $bR$ and $cR$ do not belong to $\mathcal F$. Therefore there exist units $u,v$ and irreducible elements $p_1,\ldots,p_r,q_1,\ldots,q_s$ such that:
\[
b = u p_1 \cdots p_r \text{ and } c = v q_1 \cdots q_s
\]
But this implies that
\[a = bc = \left({ uv }\right) p_1 \cdots p_r \cdot q_1 \cdots q_s
\]
which is a contradiction, since we assumed that $a$ could not be written in this form.

\paragraph{Exercise 5.} Let $R$ be a PID. Suppose we have an ascending chain of principal ideals $(a_1)\subseteq(a_2)\subseteq \ldots$ and let $I$ be the union $I=\bigcup_{i=1}^\infty (a_i)$. Obviously $I$ is an ideal, and is a principal ideal because it is in a PID. Therefore, it is generated by a single element, $I=(a)$. Since $a \in I$, $a\in (a_N)$ for some N. Then if $i\geq N$, then we have $(a)=(a_N)$, so it satisfies the ascending chain condition of principal ideals.

Let an element $a$ be irreducible. If $1\in (a)$, then $a$ would be a unit, so $(a)$ must be a proper ideal. If there is no maximal proper ideal containing $(a)$, then the ascending chain condition would not be satisfied, so we can conclude that there is a maximal ideal proper ideal I containing $(a)$ (Note: This does not require the Zorn's lemma or axiom of choice, since we did not use the theorem on maximal ideals). This ideal must be a principal ideal $(b)$ by hypothesis, but since $a\in (b)$, we have $b|a$, and since $a$ is irreducible, $b$ must either be a unit or an associate of $a$. Since $(b)$ is a proper ideal, $b$ must not be a unit, so it must be an associate of $a$. Therefore, $(a)=(b)$, so $(a)$ is maximal. However, all maximal ideals are clearly prime, so $(a)$ is a prime ideal, which implies that $a$ is prime.

\paragraph{Exercise 6.}
\begin{itemize}
\item Let $R$ be a finite integral domain. Let $a \in R$ such that $a \ne 0$. We wish to show that $a$ has a product inverse in $R$. So consider the function $f: R \to R$ defined by $f: x \mapsto a x$. We first show that the kernel of $f$ is just $\{0\}$. We have $\ker \left({f}\right) = \left\{{x \in R: f \left({x}\right) = 0}\right\} = \left\{{x \in R: a x = 0}\right\}$. Since $R$ is an integral domain, it has no zero divisors (except $0$) and thus $a x = 0$ means that $a = 0$ or $x = 0$. Since $a \ne 0$, then necessarily $x = 0$. Therefore, $\ker \left({f}\right) = \left\{{0}\right\}$ and so $f$ is injective.\\
Next, by the Pigeonhole Principle, $f$ is surjective as well. Finally, since $f$ is surjective and $1 \in R$, we have:
\[
\exists \, x \in R: f \left({x}\right) = a x = 1
\]
So $x$ is the inverse of $a$ and we are done.
\end{itemize}

\paragraph{Exercise 7.} Let $R=\Z[\sqrt{-5}]$, $\alpha=6$ and $\beta=2(1+\sqrt{-5})$. Then $\gcd(\alpha,\beta)=\emptyset$.
\begin{proof}

Define a function $N: \mathbb{Z}[\sqrt{-5}] \to \mathbb{Z}$ by 
\[
N(a+b\sqrt{-5}) = (a+b\sqrt{-5})(a-b\sqrt{-5}) = a^2+5b^2.
\]
Then:
\begin{itemize}
\item It's easy to check that $N(\alpha\beta) = N(\alpha)N(\beta)$ for all $\alpha,\beta\in\mathbb{Z}[\sqrt{-5}]$.
\item Thus, if $\alpha|\beta$ in $\mathbb{Z}[\sqrt{-5}]$, then $N(\alpha)|N(\beta)$ in $\mathbb{Z}$.
\item $\alpha\in\mathbb{Z}[\sqrt{-5}]$ is a unit if and only if $N(\alpha)=1$. In fact, $\alpha \alpha'=1 \imp N(\alpha)N(\alpha)=N(1)=1 \imp N(\alpha)=N(\alpha')=1$, and viceversa $N(\alpha)=a^2+5b^2=1 \imp b=0 \wedge a^2=1 \imp \alpha= \pm 1$.
\item Of course $a^2+5b^2 \ne 2,3$ for all $a,b \in \Z$. Thus there are no elements in $\mathbb{Z}[\sqrt{-5}]$ with $N(\alpha)=2$ or $N(\alpha)=3$.
\item It follows that $2$, $3$, $1+\sqrt{-5}$, and $1-\sqrt{-5}$ are irreducible. In fact $N(2)=4, N(3)=9$ and $N(1+\sqrt{-5})=N(1-\sqrt{-5})=6$. Suppose $2=ab$, which implies $N(a)N(b)=4$. Since the last point, this means necessarily that $N(a)=1$ or $N(b)=1$, and by the third point this means that $a$ or $b$ is a unit, i.e. $2$ is irreducible.\\
For the other elements it's sufficient to adapt the same argument.
\end{itemize}

Now suppose that $\gcd(\alpha,\beta)=\delta$ for some $\delta \in \Z[\sqrt{-5}]$. Since $\beta=2(1+\sqrt{-5})$ and $\alpha=2 \cdot 3 = (1+\sqrt{-5})(1-\sqrt{-5})$, this means that $2|\delta$ and $(1+\sqrt{-5})|\delta$, thus $2(1+\sqrt{-5})|\delta$, i.e. $\beta | \delta$. Therefore $\beta | \alpha$, i.e. $2(1+\sqrt{-5}) | 2 \cdot 3$, thus $(1+\sqrt{-5})|3$, which is not possible since $3$ is irreducible and $(1+\sqrt{-5})$ is not associate to $3$ (because the only units are $+1,-1$).


\end{proof}

\paragraph{Exercise 9.}
Let $R$ be a ring and let $aR$ be the the ideal generated by $a$. Suppose that $a = xy$ for some $x,y \in R$. Then, clearly $xy \in (a)$. So, $x \in (a)$ or $y \in (a)$, since $(a)$ is a prime ideal. 
Thus, $x = am$, or $y = an$ for some $m,n \in R$.
Since we can rewrite the last assertion as $a|x$ or $a|y$, we conclude that a is prime.\\
Viceversa, suppose that a is prime. To show that $a$ is a prime ideal, suppose that $xy \in (a)$ for some $x,y \in R$. Since $xy \in (a)$, we have that $xy = ac$ for some $c \in R$. We can rewrite this as $a|(xy)$. However, since a is prime, this implies that $a|x$ or $a|y$. So, $x = am$ or $y = an$ for some $m,n \in R$. Hence, $x \in (a)$ or $y \in (a)$, as required. 



\paragraph{An introduction to module theory.} Throughout, let $R$ be a commutative ring $A$. Submodules, factor module, and homomorphisms.
\begin{defn}
Let $M$ be an additive abelian group. An $R$-modulo structure on $M$ is a map $\sigma : R \times M \to M$, $(\lambda, x) \mapsto \lambda \cdot x$ such that for all $\lambda, \mu \in R$ and all $x,y \in M$:
\begin{itemize}
\item 
\item
\item
\end{itemize}
\end{defn}

\begin{es}\
\begin{itemize}
\item If $\lambda \in R$, then $\lambda 0 = \lambda (0+0)= \lambda 0 + \lambda 0$ and hence $\lambda 0 = 0$.
\item If $R$ is a field, then an $R$-module is an $R$-vector space.
\item $R=\Z$ every abelian group is a $\Z$-modulo (with the usual multiplication as scalar multiplication).
\item Ring multiplication: $R \times R \to R$ is an $R$-module structure, i.e. $R$ is an $R$-module.
\item Let $f: R \to S$ be a ring hom. Then $S$ is an $R$-module defined by $R \times S \to S$, $(r,s) \mapsto f(r)s$. In particular, if $R \sse S$ is a subring, then $S$ is an $R$-module by ring multiplication (e.g., $R \sse R[x_1,...,x_n])$.
\end{itemize}
\end{es}

\begin{defn}
Let $M$ be an $R$-module. A subset $N \sse M$ is called an $(R-)$submodule of $M$ if 
\begin{itemize}
\item $N \sse M$ is a subgroup
\item For all $\lambda \in R$ and all $x \in N$, $\lambda x \in N$
\end{itemize}
Then $\sigma | R \times N : R \times N \to N$ is an $R$-module structure on $N$, and $N$ is an $R$-modulo.
\end{defn}

\textbf{Remarks and examples}
\begin{itemize}
\item Let $G$ be an abelian group and $H \sse G$ a subset. Then $H \sse GG$ is a subgroup iff $H \sse G$ is a $\Z$-submodule.
\item Let $I \sse R$ be a subset. Then $I \sse R$ is an ideal iff $I \sse R$ is an $R$-submodule.
\item $0=\{0_R\}$ and $M$ are $R$-submodules of $M$. $M$ is called simple if $0 \ne M$, and $0$ and $M$ are the only submodules of $M$.
\item If $(M_\lambda)_{\lambda \in \Lambda}$ is a family of $R$-submodules, then $\bigcap_{\lambda \in \Lambda} M_\lambda$ and $\sum_{\lambda \in \Lambda} M_\lambda = \{\sum_{\lambda \in \Lambda} m_\lambda \mid m_\lambda \in M_\lambda, m_\lambda > 0$ for almost all $\lambda \in \Lambda\}$ are submodules of $M$. In particular, if $M_1$ and $M_2 \sse M$ are submodules, then $M_1+M_2=\{m_1+m_2 \mid m_1 \in M_1, m_2 \in M_2\} \sse M$ is a submodule.
\end{itemize}

\begin{defn}
Let $M$ be an $R$-module and $R \sse M$ a subset. Then
\[
_R <E>=<E>=\{\sum_{i=1}^n \lambda_i x_i \mid n \in \N, \lambda_1,...,\lambda_n \in R, x_1,...,x_n \in E\}
\]
is the submodule generated by $E$.
\end{defn}

\textbf{Remarks 1.} 
\begin{itemize}
\item Since $<E>= \bigcap_{E \sse N \sse M, N R-\text{submodule}} N = \sum_{x \in E} Rx$, $<E>$ is the smallest submodule of $M$ containing $E$.
\item If $E=\{x\}$, then $<E>=Rx$.\\
If $E=\{x_1,...,x_n\}$, then $<E>=Rx_1+...+Rx_n$.\\
If $(M_\lambda)_{\lambda \in \Lambda}$ is a family of submodules of $M$, then $<\cup_{\lambda \in \Lambda} M_\lambda > = \sum_{\lambda \in \Lambda} M_\lambda$.
\item A subset $E \in M$ is called an ($R-$module) generating set of $M$ if $_R <E>=M$. $M$ is called finitely generated if $M$ has a finite generating set.
\begin{itemize}
\item $R$ field: $M$ is a f.g. $R$-module iff $\dim_R(M)<\infty$.
\item $R=\Z$: $M$ f.g. $\Z$-module iff $M$ is a f.g. abelian group.
\item $R[X]$ is not a f.g. $R$-module (immediate).
\end{itemize}
\item Let $M$ be a f.g. $R$-module. Then every generating set contains a finite generating set.
\begin{proof}
Let $E \sse M$ be a finite generating set, and let $E' \sse M$ be an arbitrary generating set. Since $E \sse M = <E'>$, there is a finite subset $E'' \sse E'$ with $E \sse <E''>$. This implies that $M=<E> \sse <<E''>>=<E''>$, i.e. $E'' \sse E'$ is a finite generating set.
\end{proof}
\end{itemize}

\begin{defn}
Let $M$ and $N$ be $R$-modules. A map $f: M \to N$ is said to be (an $R$-module homomorphism if 
\begin{itemize}
\item $f$ is a group hom (i.e. $f(x+y)=f(x)+f(y)$).
\item $f$ is $R$-linear (i.e. $f(\lambda x)=\lambda f(x)$).\\
$\Hom_R$
\end{itemize}
\end{defn}



\end{document}