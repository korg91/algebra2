\documentclass[12pt,a4paper]{report}

\usepackage{amsmath}
\usepackage{bbm}
\usepackage[utf8]{inputenc}
\usepackage{longtable}
\usepackage{amsthm}
\usepackage{amscd}
\usepackage{amssymb}
\usepackage{amsfonts}
\usepackage{amsmath}
\usepackage{mathtools}
\usepackage{enumitem}
\usepackage[hyphens]{url}
\usepackage[scale=3]{ccicons}  % per le icone creative commons
\usepackage{hyperref}  % per i link nel pdf
\usepackage[rmargin=3.0cm,lmargin=3.0cm]{geometry}
%\usepackage{frontesp}  % prima pagina; il pacchetto frontesp.sty si trova nella stessa cartella del file .tex (deve essere adattato a mano)
\usepackage{setspace}  % per l'interlinea
\usepackage[english]{babel}  % per sillabazione
\usepackage[all]{xy} %diagrammi di funzioni
\usepackage{xspace} %per assicurare la corretta gestione degli spazi finali quando uso e.g. \AC. NB: sarebbe meglio trovare un'altra soluzione...cfr. http://tex.stackexchange.com/questions/15220/no-space-present-after-ensuremath



\theoremstyle{definition}
\newtheorem{theorem}{Theorem}[chapter] % resetta la numerazione dei teoremi per ogni capitolo
\newtheorem{corollary}[theorem]{Corollary} % la numerazione delle definizioni dipende da quella dei teoremi
\newtheorem{lemma}[theorem]{Lemma}
\newtheorem{defn}[theorem]{Definition}
\newtheorem*{addendum}{Addendum}
\newtheorem*{remark}{Remark}

%%% inizio comandi per stile per teoremi: "numero. Titolo" %%%
\newtheoremstyle{num.custom-title}
  {\topsep}   % ABOVESPACE
  {\topsep}   % BELOWSPACE
  {\normalfont}  % BODYFONT
  {0pt}       % INDENT (empty value is the same as 0pt)
  {\bfseries} % HEADFONT
  {}         % HEADPUNCT
  {5pt plus 1pt minus 1pt} % HEADSPACE
  {\thmnumber{#2.}\thmnote{ #3}}
  
\theoremstyle{num.custom-title}  
\newtheorem{teo_custom-title}[theorem]{} % per usarlo basta \begin{teo_custom-title}[<Titolo teorema>] (usa automaticamente la numerazione di [teo])
%%% fine comandi per stile per teoremi: "numero. Titolo" %%%

\newenvironment{claim}[1]{\par\noindent\underline{Claim#1:}\space}{} %per i claim
\newenvironment{claimproof}[1]{\par\noindent\underline{Proof:}\space#1}{\leavevmode\unskip\penalty9999 \hbox{}\nobreak\hfill\quad\hbox{$\blacksquare$}} %per le dimostrazioni dei claim

\DeclareMathOperator{\dom}{dom}
\DeclareMathOperator{\ran}{ran}
\DeclareMathOperator{\orb}{orb}
\DeclareMathOperator{\id}{id}
\DeclareMathOperator{\rk}{rk}
\DeclareMathOperator{\tor}{tor}
\DeclareMathOperator{\Zdv}{Zdv}
\DeclareMathOperator{\Hom}{Hom}
\DeclareMathOperator{\End}{End}
\DeclareMathOperator{\Ann}{Ann}
\DeclareMathOperator{\A}{\mathcal{A}}
\DeclareMathOperator{\B}{\mathcal{B}}
\DeclareMathOperator{\PP}{\mathcal{P}}
\DeclareMathOperator{\LL}{\mathcal{L}}
\DeclareMathOperator{\Hrtg}{\text{Hrtg}}
\DeclareMathOperator{\Ord}{\text{Ord}}
\DeclareMathOperator{\J}{\mathcal{J}}
\DeclareMathOperator{\N}{\mathbb{N}}
\DeclareMathOperator{\Q}{\mathbb{Q}}
\DeclareMathOperator{\R}{\mathbb{R}}
\DeclareMathOperator{\Z}{\mathbb{Z}}
\DeclareMathOperator{\M}{\mathfrak{M}}
\DeclareMathOperator{\U}{\mathfrak{U}}
\DeclareMathOperator{\PPP}{\mathbb{P}}
\DeclareMathOperator{\p}{\mathfrak{p}}
\DeclareMathOperator{\V}{\mathcal{V}}
\DeclareMathOperator{\a01}{\{0,1\}^{\star}}
\DeclareMathOperator{\imp}{\Rightarrow}
\DeclareMathOperator{\sm}{\setminus}
\DeclareMathOperator{\sse}{\subseteq}
\DeclareMathOperator{\Spec}{Spec}


\newcommand{\AC}{\ensuremath{\mathsf{AC}}\xspace}
\newcommand{\CC}{\ensuremath{\mathsf{CC}}\xspace}
\newcommand{\DC}{\ensuremath{\mathsf{DC}}\xspace}
\newcommand{\ZF}{\ensuremath{\mathsf{ZF}}\xspace}
\newcommand{\ZFC}{\ensuremath{\mathsf{ZFC}}\xspace}
\newcommand{\LS}{\ensuremath{\mathsf{LS}}\xspace}
\newcommand{\AMC}{\ensuremath{\mathsf{AMC}}\xspace}
\newcommand{\HRule}{\rule{\linewidth}{0.5mm}} %per la prima pagina
\newcommand{\qedblack}{\hfill $\blacksquare$}
\newcommand{\ol}{\overline}
\newcommand{\ul}{\underline}

\renewcommand{\phi}{\varphi}
\renewcommand{\S}{\mathcal{S}}
\renewcommand{\P}{\mathbb{P}}


%%%% INIZIO COMANDI PER EQUIVALENZE %%%%
\newcommand{\Implies}[2]{$\text{\ref{statement#1}}\!\implies\!\text{\ref{statement#2}}$}% X => Y
\newcommand{\punto}[1]{\item \label{statement#1}}


\newenvironment{equivalence}
    {\begin{enumerate}[label=(\arabic*),ref=(\arabic*)]
    }
    { 
	\end{enumerate}
    }
%%%% FINE COMANDI PER EQUIVALENZE %%%



% Interlinea 1.5
%\onehalfspacing  


%per le citazioni
\def\signed #1{{\leavevmode\unskip\nobreak\hfil\penalty50\hskip2em
  \hbox{}\nobreak\hfil(#1)%
  \parfillskip=0pt \finalhyphendemerits=0 \endgraf}}

\newsavebox\mybox
\newenvironment{aquote}[1]
  {\savebox\mybox{#1}\begin{quote}}
  {\signed{\usebox\mybox}\end{quote}}

%disabilita colore link
\hypersetup{%
    pdfborder = {0 0 0}
}

\begin{document}

\noindent Andrea Gadotti \hfill 01/12/2014

\paragraph{Exercise 44.} \ \\
(a)$\imp$(b): Let $\p$ be the unique prime ideal of $R$. Since every ring has a maximal ideal, and every maximal ideal is prime, $\p$ must be that maximal ideal. Thus $R/\p$ is a field. So, for any $x \in R \sm \p$, we must have $[x][y]=[1]$ for some $[y] \in R/\p$, which means that $xy \in 1+\p$. Furthermore, $\p=\sqrt{0}$, since $\sqrt{0}$ is the intersection of all the prime ideals, and $\p$ is the only one. This means that $xy=1+z$ where $z$ is nilpotent. By the fourth remark after the definition of $\J(R)$ we know that
\[
\J(R)=\{x \in R \mid 1+Rx \sse R^\times\}.
\]
But we know (immediate by definitions) that $\sqrt{0} \sse \J(R)$, so it must be true that $1+Rz \sse R^\times$, thus $1+z \in R^\times$. Therefore $xy$ is invertible, which implies that $x$ is invertible ($y(y^{-1}x^{-1})$ is the inverse of $x$).\\
(b)$\imp$(a): Suppose that every element is either a unit or nilpotent. Consider $\sqrt{0}$. $\sqrt{0}$ is maximal, since every element in $R \sm \sqrt{0}$ is a unit (and thus generates the whole ring $R$). Hence $\sqrt{0}$ is prime as well. But $\sqrt{0}$ is the intersection of all the prime ideals of $R$, and since it is maximal it must be the unique prime ideal of $R$.

\paragraph{Exercise 45.}
\begin{enumerate}
\item We know by the second remark after the definition of artinian rings that in artinian rings every prime ideal is maximal. Thus $\J(R)=\sqrt{0}$ trivially by definition.
\item Let $\Omega:=\{\mathfrak{m}_1 \cap ... \cap \mathfrak{m}_n \mid n \in \N, \mathfrak{m}_i \in \max(R)\}$. $\Omega \neq \emptyset$, since every ring has a maximal ideal. But we know by theory that a ring is artinian iff every non-empty family of ideals contains a minimal element\footnote{Proof: immediate by Zorn's lemma, ordering the family by reverse inclusion.}. Hence there exists $\mathfrak{m}_1 \cap ... \cap \mathfrak{m}_n \in \Omega$ which is minimal. Now take any maximal ideal $\mathfrak{m}$. By minimality we must have $\mathfrak{m} \cap \mathfrak{m}_1 \cap ... \cap \mathfrak{m}_n = \mathfrak{m}_1 \cap ... \cap \mathfrak{m}_n$, and therefore $\mathfrak{m}_1 \cap ... \cap \mathfrak{m}_n \sse \mathfrak{m}$. Since $\mathfrak{m}$ is prime as well, by Theorem 3.4 we obtain $\mathfrak{m}_i \sse \mathfrak{m}$ for some $i=1..n$. Since $\mathfrak{m}_i$ is maximal, this means $\mathfrak{m}_i=\mathfrak{m}$. Hence the $\mathfrak{m}_i$'s are the only maximal ideals of $R$.
\end{enumerate}






\end{document}