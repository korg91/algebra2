\documentclass[12pt,a4paper]{report}

\usepackage{amsmath}
\usepackage{bbm}
\usepackage[utf8]{inputenc}
\usepackage{longtable}
\usepackage{amsthm}
\usepackage{amscd}
\usepackage{amssymb}
\usepackage{amsfonts}
\usepackage{amsmath}
\usepackage{mathtools}
\usepackage{enumitem}
\usepackage[hyphens]{url}
\usepackage[scale=3]{ccicons}  % per le icone creative commons
\usepackage{hyperref}  % per i link nel pdf
\usepackage[rmargin=3.0cm,lmargin=3.0cm]{geometry}
%\usepackage{frontesp}  % prima pagina; il pacchetto frontesp.sty si trova nella stessa cartella del file .tex (deve essere adattato a mano)
\usepackage{setspace}  % per l'interlinea
\usepackage[english]{babel}  % per sillabazione
\usepackage[all]{xy} %diagrammi di funzioni
\usepackage{xspace} %per assicurare la corretta gestione degli spazi finali quando uso e.g. \AC. NB: sarebbe meglio trovare un'altra soluzione...cfr. http://tex.stackexchange.com/questions/15220/no-space-present-after-ensuremath



\theoremstyle{definition}
\newtheorem{teo}{Teorema}[section]  % resetta la numerazione dei teoremi per ogni capitolo
\newtheorem{defn}[teo]{Definizione}  % la numerazione delle definizioni dipende da quella dei teoremi
\newtheorem{es}[teo]{Esempio}  % idem
\newtheorem{oss}[teo]{Osservazione}  % idem
\newtheorem{prop}[teo]{Proposizione}  % idem
\newtheorem{lemma}[teo]{Lemma}  % idem
\newtheorem{corollario}[teo]{Corollario}  % idem

%%% inizio comandi per stile per teoremi: "numero. Titolo" %%%
\newtheoremstyle{num.custom-title}
  {\topsep}   % ABOVESPACE
  {\topsep}   % BELOWSPACE
  {\normalfont}  % BODYFONT
  {0pt}       % INDENT (empty value is the same as 0pt)
  {\bfseries} % HEADFONT
  {}         % HEADPUNCT
  {5pt plus 1pt minus 1pt} % HEADSPACE
  {\thmnumber{#2.}\thmnote{ #3}}
  
\theoremstyle{num.custom-title}  
\newtheorem{teo_custom-title}[teo]{} % per usarlo basta \begin{teo_custom-title}[<Titolo teorema>] (usa automaticamente la numerazione di [teo])
%%% fine comandi per stile per teoremi: "numero. Titolo" %%%


\DeclareMathOperator{\dom}{dom}
\DeclareMathOperator{\ran}{ran}
\DeclareMathOperator{\orb}{orb}
\DeclareMathOperator{\id}{id}
\DeclareMathOperator{\rk}{rk}
\DeclareMathOperator{\tor}{tor}
\DeclareMathOperator{\Zdv}{Zdv}
\DeclareMathOperator{\Hom}{Hom}
\DeclareMathOperator{\End}{End}
\DeclareMathOperator{\Ann}{Ann}
\DeclareMathOperator{\A}{\mathcal{A}}
\DeclareMathOperator{\B}{\mathcal{B}}
\DeclareMathOperator{\PP}{\mathcal{P}}
\DeclareMathOperator{\LL}{\mathcal{L}}
\DeclareMathOperator{\Hrtg}{\text{Hrtg}}
\DeclareMathOperator{\Ord}{\text{Ord}}
\DeclareMathOperator{\N}{\mathbb{N}}
\DeclareMathOperator{\Q}{\mathbb{Q}}
\DeclareMathOperator{\R}{\mathbb{R}}
\DeclareMathOperator{\Z}{\mathbb{Z}}
\DeclareMathOperator{\M}{\mathfrak{M}}
\DeclareMathOperator{\U}{\mathfrak{U}}
\DeclareMathOperator{\PPP}{\mathbb{P}}
\DeclareMathOperator{\a01}{\{0,1\}^{\star}}
\DeclareMathOperator{\imp}{\Rightarrow}
\DeclareMathOperator{\sm}{\setminus}
\DeclareMathOperator{\sse}{\subseteq}


\newcommand{\AC}{\ensuremath{\mathsf{AC}}\xspace}
\newcommand{\CC}{\ensuremath{\mathsf{CC}}\xspace}
\newcommand{\DC}{\ensuremath{\mathsf{DC}}\xspace}
\newcommand{\ZF}{\ensuremath{\mathsf{ZF}}\xspace}
\newcommand{\ZFC}{\ensuremath{\mathsf{ZFC}}\xspace}
\newcommand{\LS}{\ensuremath{\mathsf{LS}}\xspace}
\newcommand{\AMC}{\ensuremath{\mathsf{AMC}}\xspace}
\newcommand{\HRule}{\rule{\linewidth}{0.5mm}} %per la prima pagina

\renewcommand{\phi}{\varphi}
\renewcommand{\S}{\mathcal{S}}


%%%% INIZIO COMANDI PER EQUIVALENZE %%%%
\newcommand{\Implies}[2]{$\text{\ref{statement#1}}\!\implies\!\text{\ref{statement#2}}$}% X => Y
\newcommand{\punto}[1]{\item \label{statement#1}}


\newenvironment{equivalence}
    {\begin{enumerate}[label=(\arabic*),ref=(\arabic*)]
    }
    { 
	\end{enumerate}
    }
%%%% FINE COMANDI PER EQUIVALENZE %%%



% Interlinea 1.5
%\onehalfspacing  


%per le citazioni
\def\signed #1{{\leavevmode\unskip\nobreak\hfil\penalty50\hskip2em
  \hbox{}\nobreak\hfil(#1)%
  \parfillskip=0pt \finalhyphendemerits=0 \endgraf}}

\newsavebox\mybox
\newenvironment{aquote}[1]
  {\savebox\mybox{#1}\begin{quote}}
  {\signed{\usebox\mybox}\end{quote}}

%disabilita colore link
\hypersetup{%
    pdfborder = {0 0 0}
}

\begin{document}

\paragraph{Exercise 27.} 
\begin{proof}\ \\
(b)$\imp$(a) Suppose $M=M_1 \oplus M_2$. Then $e_1 \in \End_R(M)$ given by $e_1(x_1,x_2)=(x_1,0)$ is a non trivial idempotent endomorphism.\\
(a)$\imp$(b) Let $f \in \End_R(M)$ be a non-trivial idempotent endomorphism of $M$. Consider the exact sequence
\[
0 \to \ker f \hookrightarrow M \stackrel{f}{\to} f[M] \to 0.
\]
Consider the inclusion $i : f[M] \to M$. Since $f$ is idempotent, we have that $f \circ i (f(m)) = f(f(m)) = f(m)$, that is $f \circ i = \id_{f[M]}$. By splitting lemma, we obtain $M \simeq f[M] \oplus \ker f$.\\
Now we want to show that $f \neq 0,1 \imp f[M] \neq \{0\} \wedge \ker f \neq \{0\}$. If $f[M]=\{0\}$, of course $f=0$. If $\ker f = \{0\}$, then $f$ is injective. But since $f(f(m))=f(m)$ for all $m \in M$, this means that $f(m)=m$ for all $m \in M$, thus $f=1$.
\end{proof}

\paragraph{Exercise 28.} Straightforward by the hint.

\paragraph{Exercise 29.} Consider the series of the exercise. If we prove that the composition factors are $(\Z/p_1\Z, ..., \Z/p_r\Z)$, then the series is a composition series, since every $\Z/p_1\Z$ is simple because every submodule is also a subgroup, and the only subgroups of $\Z/p\Z$ with $p$ prime are trivial (since the order of any subgroup must divide $p$).\\
\\
\noindent\textbf{Claim.} For all $i$, $p_1...p_i \Z/n\Z \simeq (p_1...p_i\Z)/n\Z$, and there exists an isomorphism $\phi$ such that $\phi[p_1...p_i p_{i+1} \Z/n\Z] = (p_1...p_i p_{i+1} \Z)/n\Z$.\\
\emph{Proof of the claim.} Consider the function $\phi: p_1...p_i \Z/n\Z \to (p_1...p_i\Z)/n\Z$ given by $p_1...p_i [x] \mapsto [p_1...p_i x]$. This is trivially a well-defined isomorphism, since it's precisely the definition of scalar multiplication in the $\Z$-module $\Z/n\Z$. \qed
\\
\\
So now we have that, for all $i=\{1,...,r\}$, 
\begin{multline*}
p_1...p_i \Z/n\Z / p_1...p_i p_{i+1} \Z/n\Z \simeq (p_1...p_i \Z)/n\Z / (p_1...p_i p_{i+1} \Z)/n\Z \\
\simeq p_1...p_i \Z / p_1...p_i p_{i+1} \Z \simeq \Z/p_{i+1}\Z,
\end{multline*}
where the first ``equality'' of the second line holds thanks to the third isomorphism theorem, and the last ``equality'' holds thanks to the following isomorphism:
\[
\phi: p_1...p_i \Z / p_1...p_i p_{i+1} \Z \to \Z/p_{i+1}\Z, \ [p_1...p_i k] \mapsto [p_1...p_i k].
\]
$\phi$ is well-defined, since $[p_1...p_i k] = [p_1...p_i l]$ implies $p_1...p_i k - p_1...p_i l = p_1...p_i p_{i+1} m$ for some $m \in \Z$, which means $k-l=p_{i+1} m$, i.e. $[k]_{\Z/p_{i+1}\Z} = [l]_{\Z/p_{i+1}\Z}$. The surjectivity is totally trivial by the way $\phi$ is defined, and for the injectivity just observe that $p_1...p_i k + p_{i+1}\Z = 0 \imp p_{i+1}|k \imp p_1...p_i p_{i+1} | p_1...p_i k$.

\paragraph{Exercise 30.} Consider the following series:

\begin{multline*}
\Z_{20} \oplus \Z_{27} \supset 2\Z_{20} \oplus \Z_{27} \supset 2 \cdot 2\Z_{20} \oplus \Z_{27} \supset 2 \cdot 2 \cdot 5 \Z_{20} \oplus \Z_{27} \supset 2 \cdot 2 \cdot 5 \Z_{20} \oplus 3 \Z_{27} \\
\supset 2 \cdot 2 \cdot 5 \Z_{20} \oplus 3 \cdot 3 \Z_{27} \supset 2 \cdot 2 \cdot 5 \Z_{20} \oplus 3 \cdot 3 \cdot 3 \Z_{27} = \{0\}.
\end{multline*}
We want to prove that this is a composition series. To this aim, we just prove that the first factor is $\Z_2$, i.e. simple, since the argument can be repeated in the very same way for the other factors as well (observe that switching from a sum to the following one, one component always stays the same). Observe that, by (a very trivial application of) the second isomorphism theorem, we have:
\[
(\Z_{20} \oplus \Z_{27}) / \Z_{27} \simeq \Z_{20} \ \text{ and } \ (2\Z_{20} \oplus \Z_{27}) / \Z_{27} \simeq 2\Z_{20}.
\]
This implies
\[
\frac{\Z_{20} \oplus \Z_{27}}{2\Z_{20} \oplus \Z_{27}} \simeq \frac{(\Z_{20} \oplus \Z_{27}) / \Z_{27}}{(2\Z_{20} \oplus \Z_{27}) / \Z_{27}} \simeq \frac{\Z_{20}}{2\Z_{20}} \simeq \Z_2,
\]
where the first equality holds thanks to the third isomorphism theorem, and the last one holds because it's precisely what we proved in the second part of the previous exercise (which covers also the other cases).

\paragraph{Exercise 31.} $M$ is finitely generated by hypothesis, that is $M=Rx_1+...+Rx_k$ for some $n \in \N$ and some $x_1,...,x_k \in M$. Consider now the morphism $\Hom_R(M,N) \to N^k$ given by
\[
f \mapsto \big( f(x_1),...,f(x_n) \big).
\]
This is clearly a morphism. It is also injective since $\ker f = \{0\}$, because $\big( f(x_1),...,f(x_n) \big) = (0,...,0)$ clearly implies $f(x_i)=0$ for all $i$, which means that $f=0$. Thus $\Hom_R(M,N)$ is isomorphic to a submodule of $N^k$. But $N$ is noetherian, so $N^k$ is noetherian as well, thus finally $\Hom_R(M,N)$ is noetherian.\\


\end{document}